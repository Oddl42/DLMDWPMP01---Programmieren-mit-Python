\chapter{Introduction}

Python, benannt nach der britischen Komikertruppe Monty Python, hat sich in den vergangenen Jahren zu einem Schwergewicht unter den Programmiersprachen entwickelt, \cite{Steyer:2018}.
Sie unterstützt sowohl die objektorientierte, die prozedurale sowie die funktionale Programmierung, \cite{Häberlein:2024}. Python wurde mit dem Ziel zur Förderung eines gut lesbaren sowie knappen Programmierstiels entwickelt, \cite{Steyer:2018}. 
Im Gegensatz zu compelierten Sprachen wie Java oder C \# ist zählt Python zu den interpretierten Programmiersprachen und kann somit ohne weiteres auf einem anderen Betriebssystem ausgeführt werden, \cite{Häberlein:2024}. \\
Aufgrund der Vielzahl an Bibliotheken wie NumPy, Pandas, SciKit-Learn und MatplotLib hat Python Einzug als Standardprogramm in die Wissenschaften, die Data Science und des maschinellen Lernen erhalten, \cite{VanderPlas:2023}. \\
In der nachfolgenden Arbeit wird auf grundlegenden Techniken zur Programmentwicklung mittels Python eingegangen. Beginnend mit der Objektorientierten Programmierung, welche sich als bewährte Methode zur Erstellung komplexer Softwaresysteme erwiesen hat, \cite{Lahres:2021}.\\
Weiter werden Teile der Datenverarbeitung mit Pandas sowie dem Arbeiten mit Datenbanken behandelt. Pandas ist eine Open Source Bibliothek welche vorallem im Data Science Umfeld zum Einsatz kommt, \cite{Nelli:2023}. Eng hiermit verbunden ist das Thema des maschinellen Lernens welches ebenfalls kurz behandelt wird. Hierzu wird Scikit-Learn Bibilothek verwendet, welche effiziente Implementierungen vieler Machine Learning Algorithmen beinhaltet, \cite{Geron:2022}. Die Visualisierung erfolgt mittels \textit{bokeh}. Bokeh ist eine Python-Bibliothek zur Erstellung interaktiver Visualisierungen für moderne Webbrowser, \cite{Boekeh}\\ 
Das erzeugte Programm soll nach gängigen Entwicklungsstandards. Dies beinhaltet unter anderem auch eine klare Dokumentation. Hierzu werden sogenannte \textit{docstrings}  \textit{unittests} verwendet. Ein \textit{docstring} ist ein String-Literal welcher direkt im Sourcecode eingefügt wird, \cite{Pajankar:2022}. Ein \textit{unittest} ist eine Testmethode bei der einzelne Komponenten eines Programms unabhängig voneinander getestet werden, \cite{Pajankar:2022}. \\
Die Versionskontrolle wird über \textit{Git} verwaltet. Speziell in großen Software-Projekten bietet Git den Entwicklern die Erstellung und Pflege der Versionskontrolle und erstellt im Vergleich zu CVS per se keine kanonische Kopie der Codebasis. Alle Kopien sind Arbeitskopien und können lokal verändert werden ohne mit einem Server verbunden zu sein \cite{Russell:2019}
